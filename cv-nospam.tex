% From https://github.com/latex-ninja/hipster-cv with a lot of modifications
\documentclass[darkhipster]{hipstercv}
% available options are: darkhipster, lighthipster, pastel, allblack, grey, verylight
\usepackage[utf8]{inputenc}
\usepackage[default]{raleway}
\usepackage[margin=1cm,a4paper,includefoot,heightrounded]{geometry}
\usepackage{datenumber,fp}
\usepackage{changepage}
\usepackage{makecell}
\usepackage{url}
\usepackage{fontawesome5}
\usepackage[most]{tcolorbox}
\usepackage{xcolor}
\usepackage{xparse}
\usepackage{array}
\newcolumntype{P}[1]{>{\centering\arraybackslash}p{#1}}
\usepackage{longtable}
\usepackage{smartdiagram}

% smartdiagramset options overriden from sty package
\smartdiagramset{
    bubble center node font = \footnotesize,
    bubble node font = \footnotesize,
    % specifies the minimum size of the bubble center node
    bubble center node size = 0.5cm,
    %  specifies the minimum size of the bubbles
    bubble node size = 0.5cm,
    % specifies which is the distance among the bubble center node and the other bubbles
    distance center/other bubbles = 0.3cm,
    % sets the distance from the text to the border of the bubble center node
    distance text center bubble = 0.5cm,
    % set center bubble color
    bubble center node color = pblue!35!white,
    % define the list of colors usable in the diagram
    set color list = {materialcyan!50!white, orange!50!white, green!50!white, materialorange!50!white, materialteal!50!white, materialamber!50!white, materialindigo!50!white, materialgreen!50!white, materiallime!50!white},
    % sets the opacity at which the bubbles are shown
    bubble fill opacity = 0.6,
    % sets the opacity at which the bubble text is shown
    bubble text opacity = 1,
    description title text width=0.5cm,
    description title width=0.5cm,
    description width=5cm,
    description text width=5cm,
    descriptive items y sep =1.25,
    back arrow distance = 0,
    back arrow disabled = true,
    border color = white
}
%-------------------------------

\renewcommand{\bubblediagram}[1]{\smartdiagram[bubble diagram]{#1}}

\hypersetup{
    colorlinks,
    linkcolor={red!50!black},
    citecolor={blue!50!black},
    urlcolor={maincolor!55!black}
}

\urlstyle{sf}

% Header / footer
\usepackage{fancyhdr}
\pagestyle{fancy}
\fancyhead{} % clear all header fields
\renewcommand{\headrulewidth}{0pt} % no line in header area
\renewcommand{\footrulewidth}{0pt}
\fancyfoot{} % clear all footer fields
\fancyfoot[R]{
  \selectfont \color{black!70}
  {\small \icon{\faMapMarker}{maincolor}{} @ADDRESS@\\ \icon{\faPhone}{maincolor}{} @MOBILE@\icon{\faAt}{maincolor}{} \href{mailto:@EMAIL@}{\nolinkurl{@EMAIL@}}}
}

% \cvdegree{2015}{Docker}{Administration and Operation}{foo} \\
\renewcommand{\cvdegree}[4]{%
    {#1} & %
    \textbf{#2} \small {#3} \newline %
    {\color{black!70}\scriptsize #4} \vspace{0.5em} %
}

\newcommand{\cvsection}[1] {%
    \section*{\textbf{#1}}
}%

% -----

\definecolor{maincolor}{HTML}{0077b5}%
\definecolor{maincolorlight}{HTML}{e6f2f8}%
\definecolor{maincolorlightfont}{HTML}{000000}%
\definecolor{headerblue}{HTML}{0077b5}% override default color from template
\definecolor{headerbluefont}{HTML}{ffffff}%
\colorlet{bgskill}{lightgray!55!white}%
\colorlet{bgskillfont}{black!80!white}%

\renewcommand{\setasidefontcolour}{
    \color{black}
}

\renewcommand{\pictofraction}[2]{%
\ifthenelse{\equal{#2}{0}}{}{\pgfmathparse{#2 - 1}\foreach \n in {0,...,\pgfmathresult}{\icon{\faCircle}{#1}{\tiny}}}%
\ifthenelse{\equal{#2}{5}}{}{\pgfmathparse{5 - #2 - 1}\foreach \n in {0,...,\pgfmathresult}{\icon{\faCircle}{black!30}{\tiny}}}%
}

% \hobbyicon{\color{iconcolour}\faPlane}{Travels}{cvgreen}{\iconsize}{2em}
\renewcommand{\hobbyicon}[5]{%
    \begin{tikzpicture}%
        \draw[draw=none,fill=#3,opacity=.55] (0,0) circle (0.58);%
        \node[](icon) at (0,0) {#4#1};%
        \node[below=#5,align=center] at (icon) {#2};%
    \end{tikzpicture}
}

\renewcommand{\cvevent}[8]{%
    {\small #1 \newline #2} &
    {#3 $\cdot$ \small{{#5 ~\faMapMarker}}}\newline
    {\color{black!70}\footnotesize #6}
    \vspace{#8} &
    \raisebox{-0.7\height}{\includegraphics[height=1cm]{#7}}
}

\ExplSyntaxOn
\newcommand{\cvtags}[1] {
    \clist_map_inline:nn { #1 }
    {
        \cvtag{##1} % Apply \cvtag to each keyword
    }
}
\ExplSyntaxOff

\newcommand{\cveventdetailed}[9]{
    {\small #1} &
    {#2 $\cdot$ \small{{#4 ~\faMapMarker}}}\vspace{.5em}\newline
    {\color{black!70}\footnotesize #5\vspace{#9}} &
    {\raisebox{-0.7\height}{\includegraphics[width=2.5cm]{#6}}} \vspace{#7}
    {\par \small \cvtags{#8}}
}


\renewcommand{\header}[9]{
\tikz[remember picture,overlay] {%
\node[rectangle, fill=#6, anchor=north, minimum width=\paperwidth, minimum height=5cm](header) at (current page.north){};%
\node[left=#9 of header.north, anchor=east](name) at (header.east) {\fontfamily{\sfdefault}\selectfont #2};%
\node[anchor=south east](degree) at (name.north east) {\fontfamily{\sfdefault}\selectfont #1};%
\node[anchor=north east](descr) at (name.south east) {\fontfamily{\sfdefault}\selectfont #3};%
\node (leftqr)   [right=#7 of header.west, inner sep=0, outer sep=0] at (0,0) {\includegraphics[height=4cm]{#4}};%
\node (rightqr) [right=#8 of leftqr.west, inner sep=0, outer sep=0] at (0,0) {\includegraphics[height=4cm]{#5}};%
}%
\vspace{1.5cm}%
}

\renewcommand{\paracolbackgroundoptions}{
\backgroundcolor{c[0](4pt,4pt)(0.5\columnsep,4pt)}[HTML]{e6f2f8}
\backgroundcolor{c[1](0.5\columnsep,4pt)(4pt,4pt)}[HTML]{ffffff}
\backgroundcolor{C[0](10000pt,10000pt)(0.5\columnsep,10000pt)}[HTML]{e6f2f8}
\backgroundcolor{C[1](0.5\columnsep,10000pt)(10000pt,10000pt)}[HTML]{ffffff}
}

\renewcommand{\bg}[3]{
    \colorbox{#1}{\bfseries\color{#2}#3}
}
\newcommand{\bgtitle}[3]{%
    \begin{tikzpicture}[x=1cm,y=1cm]%
        \node at (0,0) [draw=none, fill=#1, rectangle, inner sep=0, outer sep=0, minimum width=55mm, minimum height=1.4mm]{};%
        \node at (0,0) [draw=none, rectangle, anchor=center, fill=#2]{\hspace*{0.2cm}\bfseries\large #3\hspace*{0.2cm}};%
    \end{tikzpicture}%
}
%\newcommand{\bgtitle}[3]{%
%    \begin{tikzpicture}[x=1cm,y=1cm]%
%        \node at (0,0) [draw=none, fill=#1, rectangle, left, inner sep=0, outer sep=0, minimum width=12mm, minimum height=1.4mm]{};%
%        \node at (0,0) [draw=none, rectangle, right]{\hspace*{0.2cm}\bfseries\large #3};%
%    \end{tikzpicture}%
%}
\newcommand{\bgskill}[3]{%
    %\tikz \node at (0,0) [halign=right, anchor=center, draw=none, fill=#1, rectangle, minimum width=30mm,text=#2]{#3};%
    \colorbox{#1}{\bfseries\color{#2}#3}
}
\renewcommand{\bgupper}[3]{
    \colorbox{#1}{\color{#2}\huge\bfseries\textsc{#3}}
}

% Define typographic struts, as suggested by Claudio Beccari
%   in an article in TeX and TUG News, Vol. 2, 1993.
\newcommand\Tstrut{\rule{0pt}{2.6ex}}         % = `top' strut
\newcommand\Bstrut{\rule[-0.9ex]{0pt}{0pt}}   % = `bottom' strut
\newcommand\TstrutMini{\rule{0pt}{2.3ex}}         % = `top' strut


%------------------------------------------------------------------ Variables

\newlength{\rightcolwidth}
\newlength{\leftcolwidth}
\setlength{\leftcolwidth}{0.3\textwidth}
\setlength{\rightcolwidth}{0.65\textwidth}

%------------------------------------------------------------------
\title{CV}
\author{Nicolas Maupu}
\date{September 2023}

\begin{document}

%-------------------------------------------------------------

\section*{Start}

\newcounter{dateone}%
\newcounter{datetwo}%

% Compute number of years of experience from start date
\setmydatenumber{dateone}{2007}{05}{01}%
\setmydatenumber{datetwo}{\the\year}{\the\month}{\the\day}%
\FPsub\result{\thedatetwo}{\thedateone}
\FPsub\result{\thedatetwo}{\thedateone}
\FPdiv\myage{\result}{365.2425}
\myage

\header
  {\color{headerbluefont} \huge \bfseries Nicolas Maupu}
  {\color{headerbluefont} \emph{\large SRE/DevOps, Kubernetes, Go and cloud expert}}
  {\color{headerbluefont} \emph{{\bfseries Freelance} - \FPround\myage{\myage}{0}\myage+ years of experience}}
  {qr-github.png}
  {qr-linkedin.png}
  {headerblue}
  {-0.25cm} % left margin qr 1 from header west
  {4.25cm} % left margin qr 2 relative to qr 1
  {0.7cm} % right margin


%------------------------------------------------

% The "invisible" heading has to be included here because otherwise it won't start the Paracols... strange...
\subsection*{}

\setlength{\columnsep}{0.5cm}
\columnratio{0.28}[0.7]
\begin{paracol}{2}
\hbadness=5000
\paracolbackgroundoptions

\begin{adjustwidth}{-0.5cm}{0cm} % locate left margin content a little bit more to the left

\small
{\setasidefontcolour
\bgtitle{maincolor}{maincolorlight}{About me}
\bigskip


\begin{tabular}{ll}
\faMale&Nicolas Maupu \\
\faGlobe&French \\
\faBirthdayCake&1982 \\
\faMapMarker&Paris \\
\end{tabular}

\bigskip
\bgtitle{maincolor}{maincolorlight}{Languages}
\bigskip


\begin{minipage}[t]{\leftcolwidth}
\begin{tabular}{l | ll}
\textbf{French}  & C2 & {\phantom{x}\footnotesize mother tongue} \\
\textbf{English}  & C1 &\pictofraction{maincolor}{4} \\
\end{tabular}
\end{minipage}

\bigskip
\bgtitle{maincolor}{maincolorlight}{Interests}
\bigskip

\bubblediagram{{\textbf{Cloud Expert}},  AWS, GCP, SRE/DevOPS, Azure, \textbf{Golang}, Architecture, \textbf{Kubernetes}}

%\hspace{3cm} \color{labelcolour}{OS:} \hspace{0.05em} \icon{\faLinux}{labelcolour}{\Large}
\bigskip
\bgtitle{maincolor}{maincolorlight}{Skill highlights}
\bigskip

\begin{minipage}[t]{0.3\textwidth}
\begin{tabular}{r @{\hspace{0.5em}}l}
     \multicolumn{2}{l}{\bfseries Clouds and containerization }\\
     \hline
     \bgskill{bgskill}{bgskillfont}{Kubernetes} & \pictofraction{maincolor}{4}\Tstrut\\
     \bgskill{bgskill}{bgskillfont}{Amazon AWS} & \pictofraction{maincolor}{3}\TstrutMini\\
     \bgskill{bgskill}{bgskillfont}{Google GCP} & \pictofraction{maincolor}{4}\TstrutMini\\
     \bgskill{bgskill}{bgskillfont}{Cloudflare} & \pictofraction{maincolor}{4}\TstrutMini\\
     \bgskill{bgskill}{bgskillfont}{Microsoft Azure} & \pictofraction{maincolor}{4}\TstrutMini\\[2ex]

     \multicolumn{2}{l}{\bfseries Infrastructure as Code}\\
     \hline
     \bgskill{bgskill}{bgskillfont}{Git} & \pictofraction{maincolor}{5}\Tstrut\\
     \bgskill{bgskill}{bgskillfont}{GitOps} & \pictofraction{maincolor}{4}\TstrutMini\\
     \bgskill{bgskill}{bgskillfont}{Terraform} & \pictofraction{maincolor}{4}\TstrutMini\\
     \bgskill{bgskill}{bgskillfont}{Ansible} & \pictofraction{maincolor}{4}\TstrutMini\\[2ex]

     \multicolumn{2}{l}{\bfseries Monitoring}\\
     \hline
     \bgskill{bgskill}{bgskillfont}{Prometheus} & \pictofraction{maincolor}{4}\Tstrut\\
     \bgskill{bgskill}{bgskillfont}{Thanos} & \pictofraction{maincolor}{4}\TstrutMini\\[2ex]

     \multicolumn{2}{l}{\bfseries Systems \& Languages}\\
     \hline
     \bgskill{bgskill}{bgskillfont}{Golang} & \pictofraction{maincolor}{4}\Tstrut\\
     \bgskill{bgskill}{bgskillfont}{Python} & \pictofraction{maincolor}{3}\TstrutMini\\
     \bgskill{bgskill}{bgskillfont}{Bash} & \pictofraction{maincolor}{5}\TstrutMini\\
     \bgskill{bgskill}{bgskillfont}{GNU/Linux} & \pictofraction{maincolor}{4}\TstrutMini\\[2ex]
\end{tabular}

%\dashrule{}{}
\end{minipage}

\vspace{2cm} % fill the rest of the column so that the background is stretch on the whole column (otherwise ugly white margings appear)

}
\end{adjustwidth}
%-----------------------------------------------------------
\switchcolumn

\vspace{.5em}

\begin{minipage}[t]{0.72\textwidth}
\cvsection{Profesionnal objective}
\small Kubernetes and cloud expert with 16+ years of experience in designing and managing highly available and scalable production platforms.
Looking for an opportunity to join the chess industry to ally my passion for the game and my professional life. Passionate about technology, I am excited to use all my knowledge to help you thrive.
\end{minipage}

\vspace{1.5em}

\begin{minipage}[t]{0.72\textwidth}
\cvsection{Latest experiences at a glance}
    \begin{tabular}[t]{p{1.6cm} | p{0.6\textwidth} P{0.2\textwidth} }
        \cvevent
            {Oct. 2021}{Jan. 2023}
            {\textbf{Sonepar} $\cdot$ Freelance}
            {SRE/Architect}
            {Paris (Remote)\color{cvred}}
            {
                Prepared and implemented a multi-tenants {\bfseries Kubernetes} architecture (Azure) for all Sonepar entities to allow homogeneous deployment of company's assets.
            }
            {sonepar.png}{0.8em} \\
        \cvevent
            {Aug. 2019}{Sept. 2021}
            {\textbf{e.Voyageurs SNCF} $\cdot$ Freelance}
            {SRE/Architect}
            {Lille (Remote)\color{cvred}}
            {
                Implemented an "on-demand" {\bfseries Kubernetes} stack (AWS) to conduct a smooth migration of all SNCF components to AWS.
            }
            {evoy.png}{0.8em} \\
        \cvevent
            {Nov. 2017}{Aug. 2019}
            {\textbf{Ritmx} $\cdot$ Freelance}
            {SRE/Architect}
            {Paris (half remote)\color{cvred}}
            {
                Created and led a team responsible for implementing a {\bfseries Kubernetes} infrastructure (GCP, on-prem) and led the migration of all existing Java applications to it.
            }
            {ritmx.png}{0.8em} \\
        \cvevent
            {Sept. 2016}{Oct. 2017}
            {\textbf{Mirakl} $\cdot$ Freelance}
            {SRE/Architect}
            {Paris \color{cvred}}
            {
                Developped a {\bfseries Kubernetes} platform (AWS) and migrated all applications to it with automated and continuous deployments.
            }
            {mirakl.png}{-0.8em} \\
    \end{tabular}
\end{minipage}

\vspace{1.5em}


\begin{tabular}[t]{@{}p{0.5\textwidth} c@{}} % @{} removes margins of the tabular
{
\begin{minipage}[t]{0.5\textwidth}
\cvsection{Degrees}
\begin{tabular}{r p{0.8\textwidth} c}
    \cvdegree{2007}{Master}{Analyses des Systèmes Stratégiques}{Awarded with high honours -- Versailles} \\
    \cvdegree{2003}{DUT}{G\'{e}nie informatique}{Awarded with honours -- V\'{e}lizy} \\
    \cvdegree{2001}{Bac g\'{e}n\'{e}ral}{S\'{e}rie S, sp\'{e}cialit\'{e} math\'{e}matiques}{D\'{e}ville-les-Rouen} \\
\end{tabular}

\vspace{1.5em}

\cvsection{Miscellaneous}
\small
\begin{itemize}
    \item {\bfseries{2013}} (7 months), {\bfseries{2016}} (7 months) and {\bfseries{2023}} (7 months): Travel around the world.
    \item Open source projects and experimentations:
    \begin{itemize}
        \item \emph{\bfseries{vault-secret}} creator (Kubernetes operator (Go) to inject secret resource based on Hashicorp Vault) \\ {\footnotesize \url{https://github.com/nmaupu/vault-secret}}
        \item \emph{\bfseries{freenas-provisioner}} creator (Kubernetes operator (Go) to dynamically provision a NFS volume from a freenas instance) \\ {\footnotesize \url{https://github.com/nmaupu/freenas-provisioner}}
        \item Reverse engineering work on iOt devices \\ {\footnotesize \url{https://github.com/nmaupu/yokis-hack}} \\ {\footnotesize \url{https://tinyurl.com/hackaday-yokis-hack}}
    \end{itemize}
\end{itemize}

\end{minipage}
} & {
\begin{minipage}[t]{0.19\textwidth}
\cvsection{Hobbies}
    \begin{tabular}[t]{cc}
    % usage \hobbyicon{<fontawesome icon}{Text}{background color of circle}{size of icon}{space text below icon}
    \hobbyicon{\color{iconcolour}\faPlane}{Travel}{cvgreen}{\iconsize}{2em} &
    \hobbyicon{\color{iconcolour}\faCamera}{Photo}{cvorange}{\iconsize}{2em} \\
    \hobbyicon{\color{iconcolour}\faWrench}{DIY}{cvpurple}{\iconsize}{2em} &
    \hobbyicon{\color{iconcolour}\faChessPawn}{Chess}{lightgray}{\iconsize}{2em} \\
    \hobbyicon{\color{iconcolour}\faGamepad}{Games}{red}{\iconsize}{2em} &
    \hobbyicon{\color{iconcolour}\faMountain}{Climbing}{cyan}{\iconsize}{2em} \\
    \end{tabular}
\end{minipage}
} \\

\end{tabular}

\end{paracol}



% ----- End of the main page.
% ----- Below is only detailed stuff to exclude if need be.



\cvsection{Detailed Work Experiences}
\begin{longtable}{p{1.7cm}| p{0.67\textwidth} P{0.2\textwidth}}
    \cveventdetailed
        {Oct. 2021\newline Jan. 2023}
        {Sonepar -- Freelance}
        {SRE/Architect}
        {Paris (Remote)\color{cvred}}
        {
            Prepared and implemented a multi-tenants {\bfseries Kubernetes} architecture (Azure) for all Sonepar entities to allow homogeneous deployment of company's assets.
            \begin{itemize}
                \item Automatized and deployed Kubernetes clusters (AKS) to host applications (Java micro-services and NodeJS).
                \item Massive and systematic use of the Operator Pattern \url{https://kubernetes.io/docs/concepts/extend-kubernetes/operator}.
                \begin{itemize}
                    \item Created a REST API (Go) to inventory and manage deployments of infrastructure components (using Custom Resources for storage).
                    \item Deployments managed by ArgoCD and CI/CD.
                    \item Created various operators based on this API.
                \end{itemize}
                \item Managed and resolved performance issues with related teams.
                \item Set up a log stack (Azure → Kafka → Fluentd → Splunk).
                \item Set up a highly available monitoring stack agnostic from the platform (Prometheus, Thanos, Alertmanager).
                \item Prepared multi-clouds deployment with GCP (network and agnosticity of Azure components)
            \end{itemize}
        }
        {sonepar.png}{0.5em}{Kubernetes,Azure,GCP,AKS,GKE,IaC,Golang,ArgoCD,Terraform,Velero,Prometheus,Thanos,GNU/Linux}
        {.3em} \\
    \cveventdetailed
        {Aug. 2019\newline Sept. 2021}
        {e.Voyageurs SNCF -- Freelance}
        {SRE/Architect}
        {Lille (Remote)\color{cvred}}
        {
            Implemented an "on-demand" {\bfseries Kubernetes} stack (AWS) to conduct a smooth migration of all SNCF components to AWS.
            \begin{itemize}
                \item Use of the Operator Pattern when relevant.
                \item Created a REST API (Go) to request the creation of new namespaces into clusters (DynamoDB backend).
                \item Created an operator (Go) listening this API and taking action on several events.
                \item Clusters managed as code with GitOps/FluxCD.
                \item AWS assets managed with Terraform.
            \end{itemize}
        }
        {evoy.png}{0.3em}{Kubernetes,Docker,Terraform,AWS,EKS,IaC,Golang,GitOps,FluxCD,Datadog,Velero,GNU/Linux}
        {.7em} \\
    \cveventdetailed
        {Nov. 2017\newline Aug. 2019}
        {Ritmx (SNCF company) -- Freelance}
        {SRE/Architect}
        {Paris (half remote)\color{cvred}}
        {
            Created and led a team responsible for implementing a {\bfseries Kubernetes} infrastructure (GCP, on-prem) and led the migration of all existing Java applications to it.
            \begin{itemize}
                \item Supervised project teams.
                \item Designed and implemented on-premises and GKE Kubernetes clusters using Terraform, Packer, and other tools.
                \item Prepared a unified inter-team deployment pipeline for delivering on Kubernetes.
                \item Managed secrets using HashiCorp Vault.
                \item Transitioned to Spring Cloud Config for managing application configurations.
            \end{itemize}
        }
        {ritmx.png}{1.1em}{Kubernetes,Docker,GCP,GKE,kubeadm,Golang,Terraform,Packer,Ansible,IaC,Git,GNU/Linux}
        {.7em} \\
    \cveventdetailed
        {Sept. 2016\newline Oct. 2017}
        {Mirakl -- Freelance}
        {SRE/Architect}
        {Paris\color{cvred}}
        {
            Developped a {\bfseries Kubernetes} platform (AWS) and migrated all applications to it with automated and continuous deployments.
            \begin{itemize}
                \item Implemented Kubernetes clusters to ensure the platform's future (migrating from mono-tenant to multi-tenant microservices (Java)).
                \item Containerized applications, logs handling (fluentd), etc.
                \item Deployed the entire infrastructure using Terraform and Ansible.
                \item Cultivated a DevOps best practices culture and continuous deployment awareness within teams operating in a microservices environment.
                \item Maintained operational conditions, optimizing performance, conducting profiling, and scaling as needed.
            \end{itemize}
        }
        {mirakl.png}{.15em}{Kubernetes,Docker,AWS,GCP,Terraform,Packer,Ansible,Golang,IaC,Git,GNU/Linux}
        {0em} \\
    \cveventdetailed
        {Sept. 2013\newline Feb. 2016 \newline\newline and \newline\newline May 2012 \newline Feb. 2013}
        {Orange Vallée (Orange group) -- Freelance}
        {SRE/Architect}
        {Châtillon (Haut-de-Seine)\color{cvred}}
        {
            Responsible for the technical management of the infrastructure for the Libon project at \emph{Orange Vallée}, with a specific focus on the Java/JEE backend. Two distinct roles:
            \begin{enumerate}
                \item Ensured the operational stability of the Libon application (on-call duties, handling production incidents, performance enhancements, etc.).
                \item Managed the build and strengthened the infrastructure, ensuring scalability of existing and future components, standardizing and creating sustainable inter-team processes.
            \end{enumerate}
            Led projects on the following topics:
            \begin{itemize}
                \item Architecture design for the Java/JEE components.
                \item Datacenter migration (Java/JEE foundation, system, storage, and hardware).
                \item Process industrialization and standardization (Puppet for infrastructure build and Python with Fabric framework for apps' deployment).
                \item Overhauled the monitoring/metrology system.
                \item Acted as a technical expert in the following areas: Java/JEE and JVM tuning, Puppet, SSL, storage (iSCSI SAN), GNU/Linux. Advocated for best practices in these domains.
                \item Operated in a DevOps environment and followed DevOps methodologies.
            \end{itemize}
        }
        {orange.png}{1.4em}{architecture,Apache,Tomcat,BigIP,SAN,python,puppet,foreman,Shinken,Grafana,Kibana,Elasticsearch,Git,IaC,GNU/Linux}
        {0em} \\
    \cveventdetailed
        {June 2011\newline May 2012}
        {Excilys Group}
        {SRE/Architect}
        {Cachan (Haut-de-Seine)\color{cvred}}
        {
            Managed the evolution of the outdated internal infrastructure of the Excilys Group.
            \begin{itemize}
                \item Selected the solution (VMware) and hardware (Dell m1000e blades chassis, PowerVault iSCSI SAN).
                \item Led the project and provided training on VMware and Dell hardware.
                \item Designed the architecture, conducted technical and budget validation.
                \item Implemented the solution (installation, monitoring, backups, deliveries, etc.).
            \end{itemize}
        }
        {excilys.png}{0.3em}{VMware,CentOS,Tomcat,Apache,JBoss,MySQL,Git,SAN}
        {1.5em} \\
    \cveventdetailed
        {June 2010\newline June 2011}
        {Fia-Net / Excilys consulting}
        {SRE/Architect}
        {Paris\color{cvred}}
        {
            Designed and implemented a highly available virtualized architecture to host the Kwixo application (peer-to-peer money transfers and online payments).
            \begin{itemize}
                \item Designed and implemented the architecture (Solaris 10 and zones, Apache Tomcat).
                \item Managed application delivery processes with a focus on standardization and industrialization.
                \item Deployed and configured monitoring and backups. Handled performance improvements (JVM tuning).
                \item Enacted a DevOps approach.
            \end{itemize}
        }
        {fia-net.jpg}{1.5em}{CentOS,Solaris 10,Apache,Tomcat,Java/JEE,ActiveMQ,CouchDB,Centreon,Nagios,Git}
        {1.5em} \\
    \cveventdetailed
        {May 2007\newline June 2010}
        {Excilys Group}
        {SRE/Architect}
        {Cachan (Haut-de-Seine)\color{cvred}}
        {
            Various missions, including:
            \begin{itemize}
                \item Managed Excilys and Capico (the group's e-learning application in JEE) infrastructures on an Apache stack (\emph{mod\_jk}), \emph{JBoss}, and \emph{MySQL}.
                \item Trained other Excilys group consultants on various topics (systems and networks, monitoring, virtualization, JVM profiling, Tomcat, and Apache).
                \item In 2009: mission at Weldom (Leroy Merlin group) to design and implement a highly available virtualized architecture (\emph{OpenVz}) for hosting and maintaining store order management applications.
                \item Various ad-hoc missions for analyzing JVM performance issues (HotSpot) in Java applications.
            \end{itemize}
        }
        {excilys.png}{0em}{Tomcat,Apache,JBoss,Java/JEE,JMX,OpenVz,Nagios,Zabbix,JMeter}
        {0em} \\
\end{longtable}

\end{document}
