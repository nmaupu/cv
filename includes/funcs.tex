% Allow changing letter tracking (space between letters)
\usepackage[letterspace=20]{microtype}
\def\titleFont{qag}

\usepackage{titlesec} % Allows creating custom \sections
% Format of the section titles
\titleformat{\section}
  {\fontfamily{\titleFont}\selectfont\Large\raggedright\lsstyle\bfseries\scshape}
  {}{0em}{}[\titlerule] % smallcaps, Large, continuous line - looks better if two columns, might look a bit too dramatic if just one ;)
\titlespacing{\section}{0pt}{12pt}{5pt} % Spacing around titles {<left spacing>}{<before spacing>}{<after spacing>}

% \cvdegree{2015}{Docker}{Administration and Operation}{foo} \\
\renewcommand{\cvdegree}[4]{%
  {#1} & %
  \textbf{#2} \small {#3} \newline %
  {\color{black!70}\scriptsize #4} \vspace{0.5em} %
}

\renewcommand{\pictofraction}[2]{%
  \ifthenelse{\equal{#2}{0}}{}{\pgfmathparse{#2 - 1}\foreach \n in {0,...,\pgfmathresult}{\icon{\faCircle}{#1}{\tiny}}}%
  \ifthenelse{\equal{#2}{5}}{}{\pgfmathparse{5 - #2 - 1}\foreach \n in {0,...,\pgfmathresult}{\icon{\faCircle}{black!30}{\tiny}}}%
}

% \hobbyicon{\color{iconcolour}\faPlane}{Travels}{cvgreen}{\iconsize}{2em}
\renewcommand{\hobbyicon}[5]{%
  \begin{tikzpicture}%
    \draw[draw=none,fill=#3,opacity=.55] (0,0) circle (0.58);%
    \node[](icon) at (0,0) {#4#1};%
    % vphantom is used to force the text height calculation as if there is a char like y, g, p or q present even if absent
    % Otherwise, align problem occurs when one hobby's name contains such a char and others don't...
    \node[below=#5,align=center] at (icon) {#2\vphantom{y}};%
  \end{tikzpicture}
}

\renewcommand{\cvevent}[8]{%
  {\small #1 \newline #2} &
  {#3 $\cdot$ \small{{#5 ~\faMapMarker}}}\newline
  {\color{black!70}\footnotesize #6}
  \vspace{#8} &
  \raisebox{-0.7\height}{\includegraphics[height=1cm]{#7}}
}

\ExplSyntaxOn
\newcommand{\cvtags}[1] {
  \clist_map_inline:nn { #1 }
  {
    \cvtag{##1} % Apply \cvtag to each keyword
  }
}
\ExplSyntaxOff

\newcommand{\cveventdetailed}[9]{
  {\small #1} &
  {#2 $\cdot$ \small{{#4 ~\faMapMarker}}}\vspace{.5em}\newline
  {\color{black!70}\footnotesize #5\vspace{#9}} &
  {\raisebox{-0.7\height}{\includegraphics[width=2.5cm]{#6}}} \vspace{#7}
  {\par \small \cvtags{#8}}
}

\renewcommand{\header}[9]{
  \tikz[remember picture,overlay] {%
  \node[rectangle, fill=#6, anchor=north, minimum width=\paperwidth, minimum height=5cm](header) at (current page.north){};%
  \node[left=#9 of header.north, anchor=east](name) at (header.east) {\fontfamily{\sfdefault}\selectfont #2};%
  \node[anchor=south east](degree) at (name.north east) {\fontfamily{\sfdefault}\selectfont #1};%
  \node[anchor=north east](descr) at (name.south east) {\fontfamily{\sfdefault}\selectfont #3};%
  \node (leftqr)   [right=#7 of header.west, inner sep=0, outer sep=0] at (0,0) {\includegraphics[height=4cm]{#4}};%
  \node (rightqr) [right=#8 of leftqr.west, inner sep=0, outer sep=0] at (0,0) {\includegraphics[height=4cm]{#5}};%
  }%
  \vspace{1.5cm}%
}

\renewcommand{\paracolbackgroundoptions}{
  \backgroundcolor{c[0](4pt,4pt)(0.5\columnsep,4pt)}[HTML]{e6f2f8}
  \backgroundcolor{c[1](0.5\columnsep,4pt)(4pt,4pt)}[HTML]{ffffff}
  \backgroundcolor{C[0](10000pt,10000pt)(0.5\columnsep,10000pt)}[HTML]{e6f2f8}
  \backgroundcolor{C[1](0.5\columnsep,10000pt)(10000pt,10000pt)}[HTML]{ffffff}
}

\renewcommand{\bg}[3]{
  \colorbox{#1}{\bfseries\color{#2}#3}
}
\newcommand{\bgtitle}[3]{%
  \begin{tikzpicture}[x=1cm,y=1cm]%
    \fontfamily{\titleFont}\selectfont
    \node at (0,0) [draw=none, fill=#1, rectangle, inner sep=0, outer sep=0, minimum width=55mm, minimum height=1.4mm]{};%
    \node at (0,.05) [draw=none, rectangle, anchor=center, fill=#2]{\hspace*{0.2cm}\bfseries\large\scshape\lsstyle #3\hspace*{0.2cm}};%
  \end{tikzpicture}%
}
%\newcommand{\bgtitle}[3]{%
%  \begin{tikzpicture}[x=1cm,y=1cm]%
%    \node at (0,0) [draw=none, fill=#1, rectangle, left, inner sep=0, outer sep=0, minimum width=12mm, minimum height=1.4mm]{};%
%    \node at (0,0) [draw=none, rectangle, right]{\hspace*{0.2cm}\bfseries\large #3};%
%  \end{tikzpicture}%
%}
\newcommand{\bgskill}[3]{%
  %\tikz \node at (0,0) [halign=right, anchor=center, draw=none, fill=#1, rectangle, minimum width=30mm,text=#2]{#3};%
  \colorbox{#1}{\bfseries\color{#2}#3}
}
\renewcommand{\bgupper}[3]{
  \colorbox{#1}{\color{#2}\huge\bfseries\textsc{#3}}
}

% Define typographic struts, as suggested by Claudio Beccari
%   in an article in TeX and TUG News, Vol. 2, 1993.
\newcommand\Tstrut{\rule{0pt}{2.6ex}}         % = `top' strut
\newcommand\Bstrut{\rule[-0.9ex]{0pt}{0pt}}   % = `bottom' strut
\newcommand\TstrutMini{\rule{0pt}{2.3ex}}         % = `top' strut
